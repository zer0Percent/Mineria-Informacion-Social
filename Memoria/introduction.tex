\section{Introduction}
\par Social media has become an interesting source of information for many companies since they could drive marketing plans in function of the online reputation management executed by detecting topics or analyzing the polarity of the posts related to their brand. The first step to take is to gather information related to the brand based on specific keywords that may contain valuable information. This information must be persisted taking into account legal concerns about the source information. Twitter, for example, forces not to publish the textual content of tweets. Typically, this information will be the input to machine learning algorithms since they provided a good framework to infer statistical rules. Depending of the task goal and the nature of the dataset, i.e if it is labelled or unlabelled, supervised or unsupervised algorithms must be applied. However, when dealing with social data the task is not that trivial since there exists specific lingo that complicates classification or clustering tasks. This work will assess the question of what type of information could be useful for further analysis and for these machine learning algorithms by performing an analysis on the amount of data available according to some characteristics of the datasets.

\par This document is organized as follows. The section 2 is dedicated to explain the gathering and formatting process executed during the building of the three datasets. Section 3 is focused on a high level analysis in function of common and specific variables contained in each dataset. The section 4 is dedicated to highlight the mandatory adaptations that need to be applied to the current Natural Language Processing models if social data is considered. Finally, the sections 5 and 6 are dedicated to conclusions and further work.